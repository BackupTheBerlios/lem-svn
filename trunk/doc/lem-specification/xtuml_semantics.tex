% TODO we need to be less focussed on the LEM implementation, and try
% to describe semantics in terms of a theoretical behaviour (or theoretical
% implementation).
%
% It shouldn't be much work, mainly just moving away from "in LEM, ..."

\chapter{Executable UML semantics as implemented by LEM}

\section{Scope}

This chapter rigorously defines the semantics expected of xtUML actions where
they were found to be deficient or in need of clarification. Otherwise the
semantics implemented are as described in \cite{mellor:2002}.

A rationale will be delivered in cases where behaviour significantly deviates
from that prescribed in \cite{mellor:2002}, or in cases where a significant gap
in the specification has been filled.

The action language syntax used in examples is the LEM action language, a
detailed description of which may be found in Chapter \ref{chap:actionLanguage}.

\section{Concepts and entities}
\subsection{Model error conditions}
A model error condition occurs when when the modeller has made an error in
the action language and an inconsistent or non-sensical state is reached.
Behaviour in case of a model error condition is implementation-defined.

The conditions under which a model is determined to be in an inconsistent state
are defined in later sections.

\subsubsection{Rationale}
The reason for deferring the specification of behaviour in the case of a model
error conditions is that the final implementation can determine the best way to
handle the error. For example, a mission critical redundant distributed system
may wish to immediately restart the model, while a model verifier may instead
wish to halt the interpreter and display useful error messages.

\subsection{Time}
The passage of time in an xtUML model is quantified only as intervals, and
only when used in the \textsf{delayed signal} action. The only other concept of
time is the execution of past, present future actions and state transitions,
and only then within the context of a single state machine.

\subsubsection{Rationale}
For these reasons, the only requirement for an implementation is that each
state machine have the ability for accurate measurement of time intervals, and
may be distributed without the need for a synchronised clock.

It will be defined that any action takes an amount of time $t \geq 0$ in order
to be completely executed by a state machine.

\subsubsection{Rationale}
The allowance for $t = 0$ is made to allow an action language optimiser to
completely remove a redundant or superfluous action. It would not appear to
be more constraining in practice than $t \geq \epsilon > 0$.

The allowance for arbitrarily large t is to allow an architecture for a given
model to be completely flexible with regard to its execution characteristics.
This constraint is likely to be a problem when modelling real-time systems. The
solution is for such a system to set up its own notion of time within the model
(for example, a timer tick class), however this will only be of use when
specifying and verifying the requirements of the system rather than having the
model generate a real architecture that has to run in \textsf{real time}.

It is thought that guarantees of timeliness, if any, should exist in
the architecture of implementation rather than the model.

\subsection{Objects}
\label{subsec:objects}
An object contains storage of the values of all attributes of all its
instances. These values are referred to with the same name as their
corresponding attributes.

Conceptually, the attributes of an object comprised of more than one
instance behave in the same way as would an object comprised of one
instance containing the union of the attributes of the original instances.

\codefrag{example_attributes1}{Some examples of classes with various
attributes.}{1}

Figure \ref{frag:example_attributes1} provides an illustrative example of this
concept.  The behaviour of an object comprised of instances of classes $A$ and
$B$ is identical to that of an object comprised of an instance class $C$.

LEM provides no way to resolve namespace collisions. So, in the example given,
attribute $a3$ in class $B$ may not be named $a1$ or $a2$ due to a namespace
collision with class $A$. Such a situation is a model error condition. 

An object contains the running state machines of all its instances.

In the context of any state's action language procedure, \textsf{self} is a
reference to the object in which this state machine is contained. Of course,
the object \textsf{self} may contain other state machines and instances.

\diadiagram{objects}{Example object of 2 classes}{0.42}{0}

\subsubsection{Rationale}
After examining a range of use cases, it was determined that these semantics
most closely match what is described in \cite{mellor:2002} without reducing
the expressiveness of the semantics. The semantics are simple and somewhat
similar to ``multiple inheritance'' in object oriented languages (in terms of
the contents of the object).

\subsection{Global object pool}
In LEM, the \textsf{global object pool} contains a reference to all objects
known to the LEM runtime. Objects which are not in the global object pool
cannot be selected using the \textsf{select} statement, unless otherwise
specified.

For further discussion about how objects enter and leave the global object pool,
see Sections \ref{sec:objectCreation} and \ref{sec:objectDeletion}.

\subsection{Associations}
\label{sec:relateAction}

Associations in LEM behave as defined in \cite{mellor:2002}. This includes
the relation and unrelation of objects from one another.

\subsection{State machines}
As outlined in \ref{subsec:objects}, a state machine is a property of an
instance, and a running state machine exists in the context of an object.

A state machine consumes signals and in response may transition states and
execute actions.

\subsection{Signals}
\label{subsec:signals}
A signal can be generated by a state machine and is targeted at an object.
A signal is said to be a \textsf{signal to self}, if the target is the object in
which the state machine itself is a part.

\cite{mellor:2002} describes ``external events''. These are signals which do not
originate from a state machine within the model. The generation of these signals
is implementation defined. Upon delivery, these signals behave just as any other
LEM signal.

\subsubsection{Signal delivery}
Delivery ordering of signals is as defined in \cite{mellor:2002}. That is, 2
signals from a state machine to a given object will be received in the order
they are generated. There are no other ordering requirements.

A signal delivered to an object causes the signal to be delivered to all
state machines in the object.

A signal will either be delivered to all state machines in an object, or
none at all (this is important due to signal cancellation). Note that some
or all state machines of an object may ignore any signal (this follows
from the xtUML requirements).

\subsubsection{Signal consumption}
Signal consumption ordering is as defined in \cite{mellor:2002}. That is, any
\textsf{signal to self} that has been delivered to a state machine will be
consumed before any other type of signal.

Apart from the above signal delivery semantics, state machines in the same
object are not synchronised any more than are state machines of different
objects. That is, any state machine may have consumed any number of signals
regardless of the progress of other state machines in the object.

% [signal delivery queue conceptual diagram]

\diadiagram{signalSources}{An example of two events: one from \textsf{self}, one
from another LEM object. Events to \textsf{self} are consumed before other
events and so reside on their own queue.}{0.42}{0}

\diadiagram{signalDelivery1}{After the delivery of the events in Figure
\ref{frag:signalSources}, the object has consumed the \textsf{self} event and
has delivered it to the \textsf{self} queue of each of its instances.}{0.5}{0}

\subsection{Action ordering}
Action language actions are defined to be run by a state machine, and to
change the state of the model. There may be multiple state machines running
concurrently in a model, and so there must be some prescribed ordering of
visibility of these modifications to the model as seen from another state
machine. Much like signal delivery ordering semantics must be defined.

It is defined that the results of the execution of any 2 actions by a single
state machine, in terms of visibility of changes to the model from the
perspective of an observing state machine, be ``strongly ordered'' with
respect to one another.

In other words, actions are always seen to complete in the same order that
they were issued.

To illustrate strong ordering, Figure
\ref{frag:example_concurrentAttributeManipulation} shows concurrent state
machines running sequences of attribute manipulations.  Strong ordering
specifies that 

\begin{eqnarray*}x = 20 & \rightarrow & y = 10\end{eqnarray*}

when viewed from any state machine.

Note this ordering does not only apply to attribute manipulation, but to any and
all actions which are observed from another state machine.

\sidebysidefrag{example_concurrentAttributeManipulation}
	{example_concurrentAttributeManipulation1}
	{Concurrent attribute manipulation from 2 state machines.}{1}

\subsubsection{Rationale}
The rationale for using strong action ordering is primarily that it is the
most intuitive ordering model for the action language programmer. Also, it
matches signal delivery ordering semantics as prescribed by \cite{mellor:2002}.

\section{Actions}
Actions are run by state machines. Actions provide the only mechanism to
change the current state of the model.

\subsection{Object creation}
\label{sec:objectCreation}

The create object action creates an object which is comprised of a
conglomeration of instances of the classes specified in the create action.

\codefrag{example_createAction}{A creation action with multiple classes.}{1}

The LEM code in Figure \ref{frag:example_createAction} will create an object 
comprised of instances of classes $A$,$B$ and $C$. xtUML specifies that these
classes must be participants of multiple generalisation hierarchies of the same
base class. LEM will generate an error if this restriction is violated.

When an object is created, it is added to the global object pool as part of
the the creation statement unless it has been created within an ``atomic
creation'' block. In that case, all objects created within the atomic creation
block will be atomically added to the global pool when the creation block ends.

\codefrag{example_createContext}{An example of an object creation inside an
\textsf{atomic creation} block.}{1}

For example, in Figure \ref{frag:example_createContext}, the newly created
objects will enter the global object pool at the end of the atomic creation.

\subsubsection{Rationale}
Atomic creation blocks were devised as a means to solve the problem of having
inconsistent object in the global object pool. Newly created objects often do
not meet all their invariants - the classic example is 2 objects participating
in an unconditional 1..1 relationship with one another.

Atomic creation blocks allow multiple objects to be created and made consistent
before allowing them into the wider system.

Initially, a more transparent proposal for this problem was devised -
\textsf{delayed object propagation}, in which a newly created object was not
allowed into the global object pool until its last reference disappeared. This
makes perfect sense because the object can no longer be in the process of being
set up if it cannot be referenced. The problem with this is that it could
violate the strong ``action ordering'' semantic in numerous ways. For example,
an object may not be visible to an object created after it was.

Solving this problem would have either required relaxing the action ordering,
or delaying the effects of all actions in order to match the strong action
ordering semantics. Both solutions result in a much less intuitive set of
semantics for the action language programmer.

\subsection{Object deletion}
\label{sec:objectDeletion}

Synchronous object deletion via the explicit \textsf{delete object} action is
only allowed for static objects - those without state machines. The reason it
is not allowed for objects with state machines is because it would give rise to
race conditions and special cases versus the object's state machines.

Attempted explicit deletion of a dynamic object is a model error condition.

Deletion of dynamic objects comes about when all the object's state
machines reach the deletion state. If one or more state machines reaches
a quiescent state, or otherwise never reaches a deletion state, the object
will never be deleted.

Object deletion (either static or dynamic objects), removes the object from
the global object pool.

This object may still be referred to via current references, or by following
any associations it participates in.

An object will be completely deleted when all associations are removed (or
objects participating on the other end of associations are deleted), and
when all references to the object go out of scope.

This ``delayed object deletion'' allows objects to remain in a consistent
state until all references to them disappear. 

\diadiagram{1to1}{Two classes participating in a mutually unconditional
association.}{1}{0}

Figure \ref{frag:1to1} is illustrative: an object containing an instance of A
cannot exist without being related to an object containing an instance of B.
However, deletion actions may only delete one object at a time; without delayed
object deletion, it would not be possible to delete both objects without
violating those invariants.

\subsection{Object reclassification}
The object reclassification action is removed from this xtUML specification.

\subsubsection{Rationale}
There are a number of reasons for disallowing object reclassification. The
first is that it is fundamentally very difficult to reclassify objects while
also obeying object lifetime and reference rules, and to devise a reasonable
set of semantics for the live state machines of objects being reclassified.

The burden of object lifetimes, references, and state machine consistency could
be mostly pushed into the modeller's domain however it was thought to be an
undue burden. Also, once the modeller has come to grips with these problems,
reclassification would appear to be able to be implemented by creating a new
object and deleting the old one.

The second reason is simply that no realistic models were put forward where
reclassification was a requirement. All cases appeared to be trivially
solvable using object creation and deletion.

\subsection{Attribute manipulation}
\label{sec:attributeManipulation}
An object's attributes (ie. those attributes present in all instances
that make up the object) can be directly set and queried by any state
machine that has a reference to the object.

The query and set actions are atomic with respect to other state machines.
That is, the result of an attribute query action is always guaranteed to
return the value that was provided by a previous attribute set action.

There are no other synchronisation or atomicity guarantees given. In particular
there is no possibility to perform an atomic read-modify-write operation.

It is recommended that synchronisation of attribute manipulation be
handled by signals and state machines.

% TODO: example code of an atomic incrementer state machine?

\subsection{Object selection}
\label{sec:selectAction}

The object selection action conceptually selects from all objects which are in
the global pool. Within an ``atomic create'' block, all created objects which
have not yet propagated to the global object pool will also be selected.

\subsection{Signal generation}
Signals are specified to be delivered to their target object at some time
$t >= 0$ after the signal generation action occurs, with an exception being made
for cancelled signals.

Signal delivery, ordering and consumption by target state machines is defined
in \ref{subsec:signals}.

\subsection{Delayed signals}
Delayed signals have a specified delay $d \geq 0$. Delayed signals are specified to
be delivered to their target object at time $t \geq d$ after the signal generation
action occurs, again with an exception for cancelled signals.

It is specified that a delayed signal with delay d will be delivered
\emph{after} a previously generated delayed signal with $delay \leq d$. For example,
\ref{frag:example_delayedSignals} shows a sequence of delayed signal generation,
where s2 is guaranteed to be delivered after s1.

\codefrag{example_delayedSignals}{Delayed signal generation sequence}{1}

The target object will have no knowledge of the delayed signal until it has
been delivered.

A delayed signal otherwise behaves exactly as a regular signal with respect
to delivery, ordering, consumption.

\subsection{Signal cancellation}

Signals, both immediate and delayed, to other objects may not be cancelled. The
reason for this is that a signal to another object cannot be deterministically
cancelled because there is usually no way to know whether or not that signal
has been consumed without elaborate schemes. It would probably be simpler to
solve the same problem in a different manner.

Immediate signals to self may not be cancelled. There is no use for this: the
same effect can be achieved by only delivering the signal at the point where
it was decided \emph{not} to cancel it.

Delayed signals to self are the only type of signal that may be cancelled.
A cancel signal action will delete \emph{one} signal from the object's
\textsf{self} signal queue.

% Diagram to illustrate?

A delayed signal to self may be cancelled regardless of whether the delay
time has not yet elapsed, or the delay time has elapsed and the signal
delivered (but not yet processed).

If the cancel signal action does not find a signal to delete it is a model
error with undefined results.

\textbf{Note:} When there is more than one state machine in an object, the
cancelling of delayed signals to self may be non deterministic and therefore
may result in a model error condition.

% Warning box with exclamation mark required! Just like in the textbooks!

For example, suppose an object consists of instances of classes A and B, each
with a state machine. One state machine may execute the actions shown in figure
\ref{frag:example_signalCancel}.

\codefrag{example_signalCancel}{Signal cancellation example}{1}

During this time, if the other state machine is \emph{ever} waiting for a
signal to be delivered then the cancel is non deterministic and may result in
a model error.

The reason is that the action language provides no guarantees about the
speed of execution of actions, and thus the signal delay may have already
past before the cancel statement has executed. In which case, the signal
will have been delivered and can have been picked up by the other (waiting)
state machine.

The requirement that signals are delivered to all or none of the state
machines means that the signal to self can no longer be cancelled because
it has already been delivered to one state machine.

\section{Data types}
\label{section:coreDataTypes}

LEM has two sets of types:
\begin{itemize}
	\item core types
	\item domain-specific types
\end{itemize}

\subsection{Core types}

LEM has 7 core data types:

\begin{itemize}
        \item boolean
        \item numeric
        \item date
        \item object reference
        \item set
        \item string
        \item arbitrary identifier
\end{itemize}

The semantics of each of these types are described in subsequent sections.

\subsubsection{Boolean}

The boolean type represents the values ``true'' and ``false''. Variables of
boolean type can be permuted using the familiar boolean operations: logical
``and'', ``or'' and ``not''.

\subsubsection{Numeric}

Unlike some programming languages, LEM does not allow the modeller to choose the
size of the internal representation of numeric values. Instead, LEM provides one
single type ``numeric'', which is of arbitrary size and precision.

LEM allows the modeller to place constraints on the value of numeric variables
using the domain-specific type mechanism, described in the next section.

% TODO: Add reference to domain-specific type section

If the modeller does not specify the it explicitly, an arbitrary
precision of 10 decimal places is imposed by the runtime implementation.

\subsection{Domain-specific types}

In LEM, domain-specific data types are effectively core types with constraints.

For example, the modeller may wish to specify that the price of an item is always
positive. This can be achieved with the construct shown in Figure
\ref{frag:example_DSTprice}.

\codefrag{example_DSTprice}{An example of constraining the value of a core data
type}{1}

In fact, the invariant clause can be any valid boolean expression. Any attempt
to violate the invariant will result in a \textsf{model error condition}.

